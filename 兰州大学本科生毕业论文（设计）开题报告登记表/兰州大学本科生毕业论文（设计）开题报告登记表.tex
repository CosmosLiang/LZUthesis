\documentclass{article}
\usepackage{adjustbox}
\usepackage{amsmath}
\usepackage{amssymb}
\usepackage{breqn}
\usepackage{caption}
\usepackage{xeCJK}
\usepackage{ctex}
\usepackage{float}
\usepackage{fontspec}
\usepackage{geometry}
\usepackage{graphicx}
\usepackage{indentfirst}
\usepackage{longtable}
\usepackage{multimedia}
\usepackage{multirow}
\usepackage{setspace}
\usepackage{titlesec}
\newcommand{\chuhao}{\fontsize{42pt}{\baselineskip}\selectfont} %初号
\newcommand{\xiaochuhao}{\fontsize{36pt}{\baselineskip}\selectfont} %小初号
\newcommand{\yihao}{\fontsize{28pt}{\baselineskip}\selectfont} %一号
\newcommand{\xiaoyihao}{\fontsize{24pt}{\baselineskip}\selectfont} %小一号
\newcommand{\erhao}{\fontsize{21pt}{\baselineskip}\selectfont} %二号
\newcommand{\xiaoerhao}{\fontsize{18pt}{\baselineskip}\selectfont} %小二号
\newcommand{\sanhao}{\fontsize{15.75pt}{\baselineskip}\selectfont} %三号
\newcommand{\xiaosanhao}{\fontsize{15pt}{\baselineskip}\selectfont} %小三号
\newcommand{\sihao}{\fontsize{14pt}{\baselineskip}\selectfont}% 四号
\newcommand{\xiaosihao}{\fontsize{12pt}{\baselineskip}\selectfont} %小四号
\newcommand{\wuhao}{\fontsize{10.5pt}{\baselineskip}\selectfont} %五号
\newcommand{\xiaowuhao}{\fontsize{9pt}{\baselineskip}\selectfont} %小五号
\newcommand{\liuhao}{\fontsize{7.875pt}{\baselineskip}\selectfont} %六号
\newcommand{\qihao}{\fontsize{5.25pt}{\baselineskip}\selectfont} %七号 
\setCJKfamilyfont{song}{SimSun}
\geometry{left=3cm,right=3cm,top=3cm,bottom=3cm}
\begin{document}
\pagestyle{empty}
\thispagestyle{empty}
\begin{center}
	\textbf{\sanhao{兰州大学本科生毕业论文(设计)开题报告登记表}}\\
\end{center}\xiaosihao
\begin{longtable}{|c|c|c|c|c|c|}
	\hline
	学生姓名 & 你的名字 & 性别 & 男/女 & 学号 & 32015XXXXXXX \\* \hline
	\endfirsthead
	\hline
	\endhead
	%
	\multicolumn{1}{|c|}{学\qquad 院}                                                & \multicolumn{1}{c|}{萃英学院} & \multicolumn{1}{c|}{年级}         & \multicolumn{1}{c|}{2015级} & \multicolumn{1}{c|}{专业}     & \multicolumn{1}{c|}{物理学萃英班} \\* \hline
	\multicolumn{1}{|c|}{\begin{tabular}[c]{l}指导教师\\ 姓\qquad 名\end{tabular}} & \multicolumn{1}{c|}{XXX}  & \multicolumn{1}{l|}{\begin{tabular}[c]{l}指~~导~~教~~师\\ 专业技术职务\end{tabular}} & \multicolumn{1}{c|}{教授}    & \multicolumn{1}{c|}{\begin{tabular}[c]{l}开题报告\\ 日\qquad 期\end{tabular}} & {\begin{tabular}[c]{l}~~~2018年\\ XX月XX日\end{tabular}}  \\* \hline
	\multicolumn{1}{|c|}{论文题目}                                              & \multicolumn{5}{c|}{你的论文题目}                                                                                             \\* \hline
	\multicolumn{6}{|c|}{\textbf{开题报告}}                                                                                                                                                                                                     \\* \hline
	\multicolumn{1}{|c|}{选题来源}                                              & \multicolumn{5}{c|}{$\blacksquare$基金项目 \qquad $\square$横向课题 \qquad $\square$自选 \qquad $\square$其它}                                                                                                                               \\* \hline
	\multicolumn{1}{|c|}{\begin{tabular}[c]{l}论文选题\\的意义、\\主要研究\\内容和文\\献资料调\\研情况\end{tabular}}                         & \multicolumn{5}{l|}{\begin{tabular}[l]{p{0.8\columnwidth}}该部分内容在此处完成\end{tabular}} \\* \hline
	\multicolumn{1}{|c|}{\begin{tabular}[c]{l}指导老师\\审定意见\end{tabular}}                                          & \multicolumn{5}{l|}{\begin{tabular}[c]{l}\\ \\ \\ \\ ~~~~~~~~~~~~~~~~~~~~~~~~~~~~~~~~~~~~~~~~~~~指导教师签名:\\~~~~~~~~~~~~~~~~~~~~~~~~~~~~~~~~~~~~~~~~~~~~~~~~~~~~~~~\qquad \qquad 年\qquad 月\qquad 日\end{tabular}}                                                                                                                             \\* \hline
	\multicolumn{1}{|c|}{\begin{tabular}[c]{l}教学科研\\基层组织\\或合作单\\位审定意\\见\end{tabular}}                               & \multicolumn{5}{l|}{\begin{tabular}[c]{l}\\ \\ \\ \\~~~~~~~~~~~~~~~~~~~~~~~~~~~~~~~~~~~~~~~~~~~ 负责人签名:\\~~~~~~~~~~~~~~~~~~~~~~~~~~~~~~~~~~~~~~~~~~~~~~~~~~~~~~~\qquad \qquad 年\qquad 月\qquad 日\end{tabular}}                                                                                                                              \\* \hline
	备\qquad 注                                                                      & \multicolumn{5}{l|}{\begin{tabular}[c]{l}\\ \\\end{tabular}} \\* \hline
\end{longtable}
\end{document}
